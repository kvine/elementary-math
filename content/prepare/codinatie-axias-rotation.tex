
\section{坐标变换}
\label{sec:codinatie-axias-rotation}

在高级中学数学教材中,三角函数的伸缩变换已经为人所熟知,而本文要讨论的是坐标变换的一般性理论。

这些变换包括平移、旋转、伸缩、对称,通常所说的位似变换可以通过伸缩变换得到,其中,平移、旋转、对称都是恒等变换,即不改变平面图形的形状和大小,或者说变换前后的图形是全等的,而位似变换前后的图象则是相似的,形状不变,大小可以改变。

\subsection{平移}

平移变换由平移向量$\vv{s}=(a,b)$唯一确定,平面上任意一点$P$与它在变换下的像$P'$满足$\vv{PP'}=\vv{s}$为常向量,因此如果点$P$的坐标是$(x,y)$,点$P'(x',y')$的坐标将是$(x+a, y+b)$,即
\begin{equation}
  \label{eq:shift-translation-coordinate-formula}
  \left\{
      \begin{array}{ccc}
        x' & = & x + a \\
        y' & = & y + b
      \end{array}
  \right.
\end{equation}
写成矩阵形式则是:
\begin{equation}
  \label{eq:shift-translation-coordinate-formula-matrix}
  \left(
    \begin{array}{c}
      x' \\
      y' 
    \end{array}
  \right)
  =
  \left(
    \begin{array}{c}
      x \\
      y
    \end{array}
  \right)
  +
  \left(
    \begin{array}{c}
      a \\
      b 
    \end{array}
  \right)
\end{equation}

我们把平移向量为$\vv{s}=(a,b)$的平移变换记作$S(a,b)$。

\begin{example}
  本目中来建立曲线$f(x,y)=0$在平移向量为$\vv{s}=(a,b)$的平移变换下的新方程,假使点$P(x,y)$为新曲线上任一点,则它在变换前的坐标则是$(x-a,y-b)$,而变换前的点是满足原曲线方程的,所以得到新坐标所满足的方程$f(x-a,y-b)=0$,此即原曲线在平移变换下的新的曲线的方程。比如说,以原点为圆心的圆$x^2+y^2=r^2$在此平移变换下的新方程即为$(x-a)^2+(y-b)^2=r^2$。对于一般函数$y=g(x)$而言,把它写成$f(x,y)=g(x)-y$即可应用此结论。
\end{example}

\subsection{旋转}
假如在平面直角坐标系中,有一个点的坐标是$P(x,y)$,现在我们把它绕着原点逆时针方向转动一个角度$\theta$,我们来寻求这个点的新坐标$P'=(x', y')$与原坐标之间的关系。

设向量$\overrightarrow{OP}$与坐标系$x$轴正向成角$\alpha$,则向量$\vv{OP'}$与坐标系$x$轴正向成角$\alpha+\theta$,记向量$\overrightarrow{OP}$长度为$r$,则
\begin{equation*}
  x=r\cos{\alpha},y=r\sin{\alpha}
\end{equation*}
同样有
\begin{equation*}
  x'=r\cos{(\alpha+\theta)},y'=r\sin{(\alpha+\theta)}
\end{equation*}
于是就有
\begin{equation}
  \label{eq:formulas-rotation-axias}
  \begin{split}
  x' & = x\cos{\theta} - y\sin{\theta} \\
  y' & = x\sin{\theta} + y\cos{\theta}
  \end{split}
\end{equation}
写成矩阵形式就是:
\begin{equation*}
  \left(
    \begin{array}{c}
      x' \\
      y'
    \end{array}
  \right)
    =
    \left(
      \begin{array}{cc}
        \cos{\theta} & -\sin{\theta} \\
        \sin{\theta} & \cos{\theta}
      \end{array}
    \right)
  \left(
    \begin{array}{c}
      x \\
      y
    \end{array}
  \right)
\end{equation*}

我们把这个旋转变换记作$R(\theta)$。

\begin{example}
  反比例函数$xy=1$的图象位于一三象限并且关于这两个象限的角平分线对称,我们尝试把它的图象绕原点顺时针旋转45度,看看新的曲线方程的模样。为此目的,记反比例曲线为$C$,顺时针旋转45度后得到的新曲线记为$C'$,我们反过来把$C$看成是由$C'$绕原点逆时针旋转45度得到的,就有关系式$x=\frac{\sqrt{2}}{2}(x'-y'), y=\frac{\sqrt{2}}{2}(x'+y')$,因此新曲线$C'$上的点$P(x',y')$都满足方程$\frac{\sqrt{2}}{2}(x'-y') \cdot \frac{\sqrt{2}}{2}(x'+y')=1$,即$x'^2-y'^2=2$,这是一个离心率为$\sqrt{2}$的双曲线,所以反比例函数的图象是双曲线。
\end{example}

\begin{example}
  在上例的基础上,我们来证明:函数 $y=ax+\frac{b}{x}(a \neq 0, b \neq 0)$的图象也是双曲线,为此同样期望通过旋转变换,把它变换成为双曲线的标准方程。

  首先把函数的解析式改写为以下的曲线方程
  \[ x(y-ax)=b \]
  为了方便应用前面的公式,我们设想这函数的图象是旋转后的图象,那么在旋转前的曲线的方程则是
  \[ (x\cos{\theta}-y\sin{\theta})[(x\cos{\theta}+y\cos{\theta})-a(x\cos{\theta}-y\sin{\theta})]=b \]
  把上式整理为关于 $x$ 和 $y$的二元二次方程
  \begin{equation*}
    \begin{split}
      & \cos{\theta}(\sin{\theta}-a\cos{\theta})x^2 \\
      - & \sin{\theta}(\cos{\theta}+a\sin{\theta})y^2 \\
      + & [\cos{\theta}(\cos{\theta}+a\sin{\theta})-\sin{\theta}(\sin{\theta}-a\cos{\theta})]xy \\
      = & b
    \end{split}
  \end{equation*}
  为了让它成为双曲线的标准方程,令 $xy$ 项的系数为零,即
  \[ \cos{\theta}(\cos{\theta}+a\sin{\theta})-\sin{\theta}(\sin{\theta}-a\cos{\theta}) = 0 \]
  整理得
  \[ \cos{2\theta}+a\sin{2\theta}=0 \]
  于是可取
  \[ \cos{2\theta} = \frac{a}{\sqrt{1+a^2}}, \  \sin{2\theta}=-\frac{1}{\sqrt{1+a^2}} \]
  双曲线的标准方程中除了没有 $xy$项之外,还要求$x^2$与$y^2$的系数符号相反,现在来检验一下,$x^2$项的系数是
  \[ \cos{\theta}(\sin{\theta}-a\cos{\theta})=\frac{1}{2}\sin{2\theta}-\frac{a}{2}(1+\cos{2\theta}) = -\frac{\sqrt{1+a^2}+a}{2}<0 \]
  而$y^2$的系数是
  \[ -\sin{\theta}(\cos{\theta}+a\sin{\theta})=-\frac{1}{2}\sin{2\theta}-\frac{a}{2}(1-\cos{2\theta})=\frac{\sqrt{1+a^2}-a}{2}>0 \]
  因此如果设
  \[\frac{\sqrt{1+a^2}+a}{2}=r^2, \  \frac{\sqrt{1+a^2}-a}{2}=s^2 \]
  则方程化为
  \[ -r^2x^2+s^2y^2=b^2 \]
  这正是双曲线的标准方程.
\end{example}

\subsection{伸缩}

伸缩变换是以原点为基准,把平面沿着坐标轴的两个方向进行拉伸或压缩,两个方向上各有一个 \emph{伸缩因子},分别用$\lambda(>0)$和$\mu(>0)$代表$x$和$y$方向的伸缩因子,则显然新旧坐标之间的关系是:
\begin{equation}
  \label{eq:scale-translation-coordinate-formula}
  \left\{
    \begin{array}{ccc}
      x' & = & \lambda x \\
      y' & = & \mu y
    \end{array}
  \right.
\end{equation}
矩阵形式为
\begin{equation*}
  \left(
    \begin{array}{c}
      x' \\
      y'
    \end{array}
  \right)
    =
    \left(
      \begin{array}{cc}
        \lambda & 0 \\
        0 & \mu
      \end{array}
    \right)
  \left(
    \begin{array}{c}
      x \\
      y
    \end{array}
  \right)
\end{equation*}
如果两个方向上的伸缩因子相等,则称它为以原点为中心的 \emph{位似变换}.

我们把伸缩因子分别为$\lambda$和$\mu$的伸缩变换记作$L(\lambda, \mu)$。

如果以空间的视角来看伸缩变换,它也可以看作把平面倾斜一定角度后在原平面上的投影,所以关于投影有许多跟伸缩变换完全类似的结论,比如说投影下的面积。

\begin{example}[二次曲线的离心率决定其形状]
  抛物线的标准方程$y^2=2px(p>0)$,对它进行位似变换,记原抛物线为$C$,新的曲线为$C'$,那么$C$也可以由新曲线进行位似变换得到,假定这个变换的伸缩因子为$\lambda$,就有$x=\lambda x', y=\lambda y'$,于是新的曲线方程是$\lambda^2y'^2=2p\lambda x'$,取$\lambda = p$,则得$y'^2=2x'$,这表明任何一个抛物线$y^2=2px(p>0)$的图象的形状都是一样的,不同的只是大小。事实上,同样的方法应用到圆、椭圆、双曲线上,得出的结论是: 率心率相同的二次曲线都是相似的,换句话说,离心率是确定二次曲线形状的参数,离心率相同,则形状相同,不同的只是大小。
\end{example}

\begin{example}[伸缩变换下的面积\ 椭圆的面积]
  讨论一下伸缩变换对平面图形面积的影响,根据祖暅原理,如果两个几何体在任一水平面上的截面积都相同,那么它俩体积相等。同理也有,如果两个平面图形被某一方向上的任一直线所截得的线段长度都相等,那么它俩面积也相等,进一步,如果它俩被某一方向的任意直线所截得的线段长度都有相同的比例,那么它们面积之比也为此固定比例。而伸缩变换$L(\lambda, \mu)$可以通过两次单方向的伸缩变换$L(\lambda, 1)$和$L(1, \mu)$来实现,易见前者把面积放大$\lambda$倍,后者放大$\mu$倍,所以如果一个平面图形在变换前后的面积分别为$S$和$S'$,那么就有
  \begin{equation}
    \label{eq:area-for-scale-translation}
  S'=\lambda \mu S
  \end{equation}
这是一个有趣的结论,比如说我们可以通过它得到椭圆的面积,因为标准椭圆$\frac{x^2}{a^2}+\frac{y^2}{b^2}=1(a>b>0)$在伸缩变换$L(\frac{1}{a}, \frac{1}{b})$下将变为单位圆,单位圆面积是$\pi$,所以椭圆面积就是$S=\pi a b$。
\end{example}

\begin{example}[伸缩变换下直线的斜率\ 椭圆的切线]
  \label{ep:the-tangent-after-scale-translation}
  在伸缩变换$L(\lambda, \mu)$ 下,假定一条直线$l$被变换成了直线$l'$(直线在伸缩变换下仍旧是直线这一事实,可以通过直线方程在变换之后仍旧是一个二元一次方程这一点上看出),记原直线斜率为$k$,新直线斜率为$k'$,那么对直线上两点有
  \begin{equation}
    \label{eq:tangent-of-scale-translation}
  k'=\frac{y_1'-y_2'}{x_1'-x_2'}=\frac{\mu}{\lambda}\cdot \frac{y_1-y_2}{x_1-x_2}=\frac{\mu}{\lambda}k
  \end{equation}
  因此新旧直线斜率之比是与伸缩因子有关的固定比例。有了这个结论,我们可以利用它求出椭圆曲线在任一点处的切线方程来,因为椭圆$\frac{x^2}{a^2}+\frac{y^2}{b^2}=1(a>b>0)$在伸缩变换$L(\frac{b}{a},1)$下成为圆$x^2+y^2=b^2$,因为圆的切线与过切点处的半径互相垂直,也就是斜率之积为$-1$,因此椭圆上任一点处的切线的斜率,与该点与椭圆中心连线的斜率之积为$-\frac{b^2}{a^2}$,这是一个椭圆中非常有用的结论,在椭圆上任取一点$P(x_0,y_0)$,那么可以写出过该点的切线方程
  \begin{equation*}
  y-y_0=-\frac{b^2x_0}{a^2y_0}(x-x_0)
  \end{equation*}
  利用$\frac{x_0^2}{a^2}+\frac{y_0^2}{b^2}=1$即可将它化简为
  \begin{equation*}
  \frac{x_0x}{a^2}+\frac{y_0y}{b^2}=1
  \end{equation*}
\end{example}

\begin{example}
  如果允许伸缩因子是虚数,那么在伸缩变换$L(1,i)$下($i$是虚数单位),双曲线将被变换为椭圆! 然而这对现阶段的我们似乎也没有什么用处。
\end{example}

\begin{example}
  一对斜率互为相反数的直线分别与同一个标准椭圆(中心在坐标原点,长短轴平行于坐标轴)相交得四个交点,证明:这四个交点共圆.

  证明过程将会用到伸缩变换,先来讨论下在伸缩变换下,线段的长度变化情况,在伸缩变换 $L(\lambda,\mu)$下,一条变换前长度为$l$,所在直线的斜率为$k$的线段,不难证明,在变换后的长度是
  \[ l'=l \sqrt{\frac{\lambda^2+\mu^2k^2}{1+k^2}} \]
  即长度的变化与该线段的倾斜程度有关,但对于这里问题中的两条斜率互为相反数的直线来说,位于它们上的两条线段在作伸缩变换时,长度变化比例是相同的。

  设椭圆长轴长为 $2a$,短轴长为 $2b$,我们通过一个伸缩变换 $L\left(\frac{1}{a},\frac{1}{b}\right)$,将椭圆变换为单位圆,此时两条直线仍然是两条斜率互为相反数的直线(但是斜率发生了变化),这两条直线必然相交,如果交点位于椭圆内部,则变换后则位于单位圆内部,分割出来的四条线段满足相交弦定理的乘积式,然后我们作逆变换回去$L(a,b)$,此时由于这四条线段的长度变化比例相同,所以在椭圆中,这四条线段仍然满足相交弦定理的乘积式,由相交弦定理的逆定理,这四点共圆。

  如果两条直线的交点在椭圆外,那么只要把上面的相交弦定理改成割线定理,即可知结论仍然成立.
\end{example}

还有一种变换称为 \emph{反射变换},也称为对称变换,即把平面上所有的点关于某一直线作对称,如果该对称轴与坐标轴成倾斜角度,可以先做旋转变换,使得对称轴与坐标轴平行或者垂直,再通过平移使对称轴与坐标轴重合,这时只需要关于坐标轴作对称之后再反平移和反旋转,即可实现一般的反射变换,所以只要讨论关于坐标轴的反射变换就可以了,然而这一点如果允许伸缩变换中的伸缩因子为负就可以得到,所以反射变换本质上是平移、旋转和伸缩的叠加,就不单独讨论了。

\subsection{错切变换}
\label{sec:shear-translation}

错切变换是一种倾斜变换,把一个矩形变换成一个平行四边形,想象一下,矩形的四个顶点分别是$(0,0),(2,0),(2,1),(0,1)$,然后在横轴上的长边不动,让在直线$y=1$上的长边在该直线上左右移动一段距离$d$,那么得到一个平行四边形,其四个顶点是$(0,0),(2,0),(2,1+d),(0,1+d)$,这实际是把矩形经过了拉扯得出来的,矩形原来的两条宽,本来是与$x$轴垂直的,但变换之后,成了两条与$x$轴成一定角度的倾斜线段。

那么,考虑在这样的变换下,平面上任意一点的新旧坐标关系,首先纵坐标是没有变化的,变化的是横坐标,横坐标在原来的基础上做了一个偏移,但是这个偏移量,是跟纵坐标成比例的,设比例因子为$u$,那么得到坐标变换公式
\begin{equation}
  \label{eq:scale-translation-coordinate-formula}
  \left\{
    \begin{array}{ccc}
      x' & = &  x + uy \\
      y' & = &  y
    \end{array}
  \right.
\end{equation}
矩阵形式为
\begin{equation*}
  \left(
    \begin{array}{c}
      x' \\
      y'
    \end{array}
  \right)
    =
    \left(
      \begin{array}{cc}
        1 & u \\
        0 & 1
      \end{array}
    \right)
  \left(
    \begin{array}{c}
      x \\
      y
    \end{array}
  \right)
\end{equation*}

这是对横坐标做比例偏移的结果,那么同样我们也可以对纵坐标做比例,相当于上面矩形的宽边保持不动,但把两条长边变成倾斜线段,也就是做纵向的拉扯,设对应的比例因子为$v$,那么同样可以得到变换公式
\begin{equation}
  \label{eq:scale-translation-coordinate-formula}
  \left\{
    \begin{array}{ccc}
      x' & = &  x \\
      y' & = &  vx+y
    \end{array}
  \right.
\end{equation}
矩阵形式为
\begin{equation*}
  \left(
    \begin{array}{c}
      x' \\
      y'
    \end{array}
  \right)
    =
    \left(
      \begin{array}{cc}
        1 & 0 \\
        v & 1
      \end{array}
    \right)
  \left(
    \begin{array}{c}
      x \\
      y
    \end{array}
  \right)
\end{equation*}
这就是在两个方向分别做比例偏移的错切变换的坐标变换公式,如果我们把这两个方向上的变换综合起来,在横方向上作比例因子为$u$的变换,在纵方向上作比例因子为$v$的变换,那么,新的变换矩阵实际上只要把两个变换矩形相乘即可,如果是先作横向变换,再作纵向变换,那么变换公式是
\begin{equation*}
  \left(
    \begin{array}{c}
      x' \\
      y'
    \end{array}
  \right)
    = 
    \left(
      \begin{array}{cc}
        1 & 0 \\
        v & 1
      \end{array}
    \right)
\cdot
    \left(
      \begin{array}{cc}
        1 & u \\
        0 & 1
      \end{array}
    \right)
 \cdot
  \left(
    \begin{array}{c}
      x \\
      y
    \end{array}
  \right)
  \\
   = 
     \left(
      \begin{array}{cc}
        1 & u \\
        v & uv+1
      \end{array}
    \right)
 \cdot
  \left(
    \begin{array}{c}
      x \\
      y
    \end{array}
  \right)
\end{equation*}
但如果是先作纵向变换,再作横变换,则变换公式是
\begin{equation*}
  \left(
    \begin{array}{c}
      x' \\
      y'
    \end{array}
  \right)
    = 
    \left(
      \begin{array}{cc}
        1 & u \\
        0 & 1
      \end{array}
    \right)
\cdot
    \left(
      \begin{array}{cc}
        1 & 0 \\
        v & 1
      \end{array}
    \right)
 \cdot
  \left(
    \begin{array}{c}
      x \\
      y
    \end{array}
  \right)
  \\
   = 
     \left(
      \begin{array}{cc}
        1+uv & u \\
        v & 1
      \end{array}
    \right)
 \cdot
  \left(
    \begin{array}{c}
      x \\
      y
    \end{array}
  \right)
\end{equation*}
可见,这两次变换的顺序会影响最终的结果。


\subsection{一般线性变换}
\label{sec:two-dimension-linear-translation}

前面讨论了几种形式的变换,它们都是最基本的几种\emph{线性变换},而一般的线性变换是
\begin{equation*}
  \left\{
    \begin{array}{ccc}
      x' & = &  a_{11}x+a_{12}y+b_1 \\
      y' & = &  a_{21}x+a_{22}y+b_2
    \end{array}
    \right.
  \end{equation*}
  记矩阵
  \[ \bm{A} =
     \left(
      \begin{array}{cc}
        a_{11} & a_{12} \\
        a_{21} & a_{22}
      \end{array}
    \right),
    \bm{b}=
    \left(
      \begin{array}{c}
        b_1 \\
        b_2
      \end{array}
      \right)
  \]
  那么这坐标变换的矩阵形式就是
  \[ \bm{x'} = \bm{A}\bm{x}+\bm{b}  \]
  其中
  \[ \bm{x'}=
    \left(
      \begin{array}{c}
        x' \\
        y'
      \end{array}
    \right),
    \bm{x} =
    \left(
      \begin{array}{c}
        x \\
        y
      \end{array}
    \right)
  \]
矩阵形式的变换式简洁,其形式体现出了\emph{线性}的真正含义,就是原坐标与常量的线性组合。

接下来要讨论的问题是,对于一个一般形式的线性变换,要将其分解为几个基本变换的复合。



%%% Local Variables:
%%% mode: latex
%%% TeX-master: "../../elementary-math-note"
%%% End:
